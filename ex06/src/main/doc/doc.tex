\documentclass[a4paper,headings=small]{scrartcl}
\KOMAoptions{DIV=12}

\usepackage[utf8x]{inputenc}
\usepackage{amsmath}
\usepackage{graphicx}
\usepackage{multirow}
\usepackage{listings}
\usepackage{subfigure}

% define style of numbering
\numberwithin{equation}{section} % use separate numbering per section
\numberwithin{figure}{section}   % use separate numbering per section

% instead of using indents to denote a new paragraph, we add space before it
\setlength{\parindent}{0pt}
\setlength{\parskip}{10pt plus 1pt minus 1pt}

\title{Expectation Maximization 2}
\subtitle{Excercise 6 \\ Automatic Image Analysis - WS12/13}
\author{\textbf{Team E}: Marcus Grum (340733), Robin Vobruba (343773), \\ Robert Koppisch (214168), Nicolas Brieger (318599)\\Carola Thrams (314681)}
\date{\today}

\pdfinfo{%
	/Title    (Automatic Image Analysis - WS12/13 - Excercise 6 - Expectation Maximization 2)
	/Author   (Team E: Marcus Grum (340733), Robin Vobruba (343773), Robert Koppisch (214168), Nicolas Brieger (318599), Carola Thrams (314681))
	/Creator  ()
	/Producer ()
	/Subject  ()
	/Keywords ()
}

% Simple picture reference
%   Usage: \image{#1}{#2}{#3}
%     #1: file-name of the image
%     #2: percentual width (decimal)
%     #3: caption/description
%
%   Example:
%     \image{myPicture}{0.8}{My huge house}
%     See fig. \ref{fig:myPicture}.
\newcommand{\image}[3]{
	\begin{figure}[htbp]
		\centering
		\includegraphics[width=#2\textwidth]{#1}
		\caption{#3}
		\label{fig:#1}
	\end{figure}
}
\newcommand{\imgRoot}{../resources/img}
\newcommand{\imgGeneratedRoot}{../../../target/pics}

\begin{document}

\maketitle

\section{Results:}

As expected, the result of a good Confusion Matrix shows mostly high diagonal elements,
as can be seen in Fig. \ref{fig:\imgGeneratedRoot/ConfusionMatrix_20PCA_10Cluster.jpg}.
This means, that the number that was chosen in the reality - x-axis - has correctly chosen
by our algorithm - y-axis.
Our algorithm has - based on 20 Principal Components and 10 Clusters
as they have been used in Fig. \ref{fig:\imgGeneratedRoot/ConfusionMatrix_20PCA_10Cluster.jpg} -
correctly detected 946 numbers.

\image{\imgGeneratedRoot/ConfusionMatrix_20PCA_10Cluster.jpg}{0.8}{%
		resulting good Confusion Matrix.}

In Fig.~\ref{fig:worseConfusionMatrices}, you can see the results for two worse Confusion Matrices.
Logically, the use of one Cluster and only one Principal Component is worse (left side).
It only has detected 43 numbers correctly.
A bit surprising has been the worse result of the maximum of clusters and principal components 
of our consideration (right side). The use of 30 clusters and 30 prinsipal components only detected 99 numbers
correctly. In our explanation, this is because of our EM-algorithm. In this region, the 
values become numerically instable and the fitting becomes wrong.

\begin{figure}
\subfigure[1 PCAs and 1 Cluster ]{\frame{\includegraphics[width=0.49\textwidth]{\imgGeneratedRoot/ConfusionMatrix_1PCA_1Cluster.jpg}}} \hfill
\subfigure[30 PCAs and 10 Cluster]{\frame{\includegraphics[width=0.49\textwidth]{\imgGeneratedRoot/ConfusionMatrix_30PCA_10Cluster.jpg}}} \hfill
\caption{resulting worse Confusion Matrices.}
\label{fig:worseConfusionMatrices}
\end{figure}

Our results have been visualized in Fig. \ref{fig:\imgGeneratedRoot/PerformanceAnalysation.jpg}.
For this, we have generated numbers of the grid containing elements for 1, 3, 7 and 10 clusters
and for 1, 5, 10, 20 and 30 principal components. The numbers in between the grid elements
have been interpolated linearly.

\image{\imgGeneratedRoot/PerformanceAnalysation.jpg}{0.8}{%
		resulting performance.}

In the figure, one can see that the results of correctly classified numbers increases with increasing
cluster and increasing principal components untill the very top right.
At this corner, the classification correctness falls immeadetly. 
For this, we have given an explanation above. 
The question remains, why the result given in the exercise presentation do not show such a drop...?
In our explanation, this is because we have chosen other grid elements.
If we had chosen the 25th PCA element as last grid element and had interpolated the missing
numbers thill the 30th element, we would not have detected such a drop.

\newpage
\section{Printed Code C++:}

\lstinputlisting[breaklines=true,language=C++]{../native/aia6.cpp}

\newpage
\section{Printed Code Matlab-Visualisierung:}

\lstinputlisting[breaklines=true,language=C++]{../native/Visualisierung.m}

\end{document}
