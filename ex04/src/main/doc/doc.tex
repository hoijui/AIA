\documentclass[a4paper,headings=small]{scrartcl}
\KOMAoptions{DIV=12}

\usepackage[utf8x]{inputenc}
\usepackage{amsmath}
\usepackage{graphicx}
\usepackage{multirow}
\usepackage{listings}
\usepackage{subfigure} 

% define style of numbering
\numberwithin{equation}{section} % use separate numbering per section
\numberwithin{figure}{section}   % use separate numbering per section

% instead of using indents to denote a new paragraph, we add space before it
\setlength{\parindent}{0pt}
\setlength{\parskip}{10pt plus 1pt minus 1pt}

\title{Naive Bayes}
\subtitle{Excercise 4 \\ Automatic Image Analysis - WS12/13}
\author{\textbf{Team E}: Marcus Grum (340733), Robin Vobruba (343773), \\ Robert Koppisch (214168), Nicolas Brieger (318599)\\Carola Thrams (314681)}
\date{\today}

\pdfinfo{%
  /Title    (Automatic Image Analysis - WS12/13 - Excercise 4 - Naive Bayes)
  /Author   (Team E: Marcus Grum (340733), Robin Vobruba (343773), Robert Koppisch (214168), Nicolas Brieger (318599), Carola Thrams (314681))
  /Creator  ()
  /Producer ()
  /Subject  ()
  /Keywords ()
  %Version 1
}

% Simple picture reference
%   Usage: \image{#1}{#2}{#3}
%     #1: file-name of the image
%     #2: percentual width (decimal)
%     #3: caption/description
%
%   Example:
%     \image{myPicture}{0.8}{My huge house}
%     See fig. \ref{fig:myPicture}.
\newcommand{\image}[3]{
	\begin{figure}[htbp]
		\centering
		\includegraphics[width=#2\textwidth]{#1}
		\caption{#3}
		\label{fig:#1}
	\end{figure}
}
\newcommand{\generatedImgRoot}{../resources/img}
\newcommand{\generatedImgRootTwo}{../../../target}

\begin{document}

\maketitle


\section{Naive Bayes - Pixel-wise Object Classification with OpenCV}


\subsection{The Task}
In this exercise a pixel-wise object classification with help of Naive Bayes is realized.
The task consisted of detecting multiple objects in any test image, 
such that they are colored w.r.t. their classification. 
This will done based on one train image, that contains n classification objects.
For this, an input picture has been given that contains several interesting objects and 
quite a lot disturbance objects. The whole situation can be seen in 
fig. \ref{fig:\generatedImgRoot/Input.png}.

\image{\generatedImgRoot/Input.png}{0.9}{%
		Training picture (shows 3 categorization objects).}
\newpage
In this picture, you see three cars. The left one is a black car,
the car in the center is a red one and the right car is a white one.
Additionally, one can find there a smooth blue sky, some clouds and a white floor.
Far away in the background, on can see the conours of a town.

Each classification is introduced by two reference images. 
The first one shows the specific classification object in white,
the second one shows the non-classification objects as the background in black.
The two corresponding pictures of the black car classification are shown in fig. ~\ref{fig:label1}.
The two corresponding pictures of the red car classification are shown in fig. ~\ref{fig:label2}.
The two corresponding pictures of the white car classification are shown in fig. ~\ref{fig:label3}.
Because of the aim to find cars with ferrari potential, this classification is called
the ferrari potential problem and is explained in a later chapter.

\begin{figure}
\subfigure[black car classification]{\frame{\includegraphics[width=0.33\textwidth]{\generatedImgRoot/schwarzes_Auto_weiss.png}}} \hfill
\subfigure[background classification]{\frame{\includegraphics[width=0.33\textwidth]{\generatedImgRoot/schwarzes_Auto_schwarz.png}}}
\caption{reference images of black car classification}
\label{fig:label1}
\end{figure}
\begin{figure}
\subfigure[red car classification]{\frame{\includegraphics[width=0.33\textwidth]{\generatedImgRoot/rotes_Auto_weiss.png}}}\hfill
\subfigure[background classification]{\frame{\includegraphics[width=0.33\textwidth]{\generatedImgRoot/rotes_Auto_schwarz.png}}}
\caption{reference images of black car classification}
\label{fig:label2}
\end{figure}
\begin{figure}
\subfigure[white car classification]{\frame{\includegraphics[width=0.33\textwidth]{\generatedImgRoot/weisses_Auto_weiss.png}}}\hfill
\subfigure[background classification]{\frame{\includegraphics[width=0.33\textwidth]{\generatedImgRoot/weisses_Auto_schwarz.png}}}
\caption{reference images of black car classification}
\label{fig:label3}
\end{figure}

\subsection{Interim Test}
As a kind of a first interim test, we also took the training as test image,
as one can see in fig. ~\ref{fig:label5} based on the given normRGB(...) function.

\begin{figure}
\subfigure[black car classification]{\frame{\includegraphics[width=0.33\textwidth]{\generatedImgRootTwo/result_ml_RGB_weiss.png}}}\hfill
\subfigure[red car classification]{\frame{\includegraphics[width=0.33\textwidth]{\generatedImgRootTwo/result_ml_RGB_rot.png}}}
\subfigure[white car classification]{\frame{\includegraphics[width=0.33\textwidth]{\generatedImgRootTwo/result_ml_RGB_schwarz.png}}}
\caption{Naive Bayes result on training image - RGB}
\label{fig:label5}
\end{figure}

Here, one can see in picture b) that the red parts of the car are quite well detected again.
The black car is detected as well because in the black wheels of the car in the 
red reference picture have been selected as well what causes the detection of the black car.
Corresponding parts of the white care are detected as well. 
In picture a), one can see that the reflection of the sun on the black car causes a huge
detection of the floor as well as major parts of the white car.
In picture c), one can see that black parts of the white care cause a detection of the 
entire black car and some black parts of the red and white car. 
The white car itself is not detected properly maybe because of the interlance of the
very similar floor that has fallen to the background cluster.

As a kind of a second interim test, we also took the training as test image,
as one can see in fig. ~\ref{fig:label6} based on the given hsv(...) function.

\begin{figure}
\subfigure[black car classification]{\frame{\includegraphics[width=0.33\textwidth]{\generatedImgRootTwo/result_ml_HSV_weiss.png}}}\hfill
\subfigure[red car classification]{\frame{\includegraphics[width=0.33\textwidth]{\generatedImgRootTwo/result_ml_HSV_rot.png}}}
\subfigure[white car classification]{\frame{\includegraphics[width=0.33\textwidth]{\generatedImgRootTwo/result_ml_HSV_schwarz.png}}}
\caption{Naive Bayes result on training image - HSV}
\label{fig:label6}
\end{figure}

Those should just give a further impression for a second feature calculation method.

As a kind of a third interim test, we also took the training as test image, 
combining the given normRGB(...) and the given hsv(...) function.
Butregrettably, herefor the RAM of our computers was not sufficient.

In deed, a combination, some new feature detection methods would have been very interesting,
but so far we were happy with the results of the normRGB(...) and continued.

\subsection{Ferrari Potential Problem ;-)}
With help of the Naive Bayes, we aime to classify baby cars, that are still very young.
While they are becoming older, they might change into a ferrari when its owner drives passionatly.
Given the 3 reference classifications, we try to estimate if a black car can change into a ferrari,
if a red car can change into a ferrari, or if a white car can change into a ferrari.

As a test image, of course a real ferrari has been taken, 
as one can see in fig. \ref{fig:\generatedImgRoot/test_image1.jpeg}.

\image{\generatedImgRoot/test_image1.jpeg}{0.8}{%
		Test picture (shows the target object - a real ferrari).}

Here, one can see a red ferrari that is standing on a white floor before a white background.

\section{Results:}

As a result, we detected that only the red car classification classified the red ferrari correctly.
This is because of the high ferrari potential of a red VW (similar red color)
and can be seen in fig. \ref{fig:\generatedImgRootTwo/1rot_result_ml.png}.

\image{\generatedImgRootTwo/1rot_result_ml.png}{0.8}{%
		Naive Bayes result on test picture - RGB (based on red car classification).}

Here, one can see that the area of the ferrari 
that was shown in fig. \ref{fig:\generatedImgRoot/test_image1.jpeg}
has been nearly completly detected and correctly marked in red.
The background classification is also quite good separated in green,
at least on base of our visual impression.

For a comparison, the result of Naive Bayes on base of the black and green car classification
is shown in fig. ~\ref{fig:label4}.

\begin{figure}
\subfigure[black car classification]{\frame{\includegraphics[width=0.49\textwidth]{\generatedImgRootTwo/1schwarz_result_ml.png}}} \hfill
\subfigure[green car classification]{\frame{\includegraphics[width=0.49\textwidth]{\generatedImgRootTwo/1weiss_result_ml.png}}}
\caption{Naive Bayes result on test picture - RGB}
\label{fig:label4}
\end{figure}

Here, one can see that the neither the black VW nor the green VW have a high ferrari potential,
as they are not correctly colored in red in the pictures.

\section{Interpretation:}

Because of the fact, that we have used in our implementation only the normRGB(...) function,
the red car classification can be seen as ferrari classification in our case.
We have chosen a picture of a red ferrari because of the cliche that ferraris are red.
The only reason, why the ferrari has been detected with help of Naive Bayes,
has been the luckily situation, that one of the three classification cars has had
the similar red color as the ferrari had.
Indeed, every other ferrari picture, e.g. of a yellow one, would have had a similar bad
classification result concerning the red car reference picture 
as the black and white reference pictures had w.r.t. the shown red ferrari.

\newpage
\section{Printed Code:}

%\lstset{language=<C++>}
%\begin{lstlisting}
%test
%\end{lstlisting}
\lstinputlisting[breaklines=true]{../native/aia4.cpp}

\end{document}