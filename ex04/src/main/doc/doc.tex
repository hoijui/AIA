\documentclass[a4paper,headings=small]{scrartcl}
\KOMAoptions{DIV=12}

\usepackage[utf8x]{inputenc}
\usepackage{amsmath}
\usepackage{graphicx}
\usepackage{multirow}
\usepackage{listings}
\usepackage{subfigure}

% define style of numbering
\numberwithin{equation}{section} % use separate numbering per section
\numberwithin{figure}{section}   % use separate numbering per section

% instead of using indents to denote a new paragraph, we add space before it
\setlength{\parindent}{0pt}
\setlength{\parskip}{10pt plus 1pt minus 1pt}

\title{Naive Bayes}
\subtitle{Excercise 4 \\ Automatic Image Analysis - WS12/13}
\author{\textbf{Team E}: Marcus Grum (340733), Robin Vobruba (343773), \\ Robert Koppisch (214168), Nicolas Brieger (318599)\\Carola Thrams (314681)}
\date{\today}

\pdfinfo{%
  /Title    (Automatic Image Analysis - WS12/13 - Excercise 4 - Naive Bayes)
  /Author   (Team E: Marcus Grum (340733), Robin Vobruba (343773), Robert Koppisch (214168), Nicolas Brieger (318599), Carola Thrams (314681))
  /Creator  ()
  /Producer ()
  /Subject  ()
  /Keywords ()
  %Version 1
}

% Simple picture reference
%   Usage: \image{#1}{#2}{#3}
%     #1: file-name of the image
%     #2: percentual width (decimal)
%     #3: caption/description
%
%   Example:
%     \image{myPicture}{0.8}{My huge house}
%     See fig. \ref{fig:myPicture}.
\newcommand{\image}[3]{
	\begin{figure}[htbp]
		\centering
		\includegraphics[width=#2\textwidth]{#1}
		\caption{#3}
		\label{fig:#1}
	\end{figure}
}
\newcommand{\imgRoot}{../resources/img}
\newcommand{\imgGeneratedRoot}{../../../target}

\begin{document}

\maketitle


\section{Naive Bayes - Pixel-wise Object Classification with OpenCV}


\subsection{The Task}
In this exercise a pixel-wise object classification with help of Naive Bayes is realized.
The task was to detect multiple objects in a test image,
such that they are colored w.r.t. to their classification.
This is done based on a training image, that contains $n$ classification objects.
For this, an input picture was chosen that contains several objects of interest as well as disturbance objects (see fig. \ref{fig:\imgRoot/Input.png}).

\image{\imgRoot/Input.png}{0.9}{%
		Training picture (shows 3 categorization objects).}
\newpage
In this picture three cars with differing colors, standing on a white floor, can be seen: Red, black and white.
Additionally, the picture consists of a smooth portion portion of the blue sky and one that is disturbed by clouds.
Far away in the background one can see the conours of a town.

Each classification is introduced by two reference images.
The first of each series shows the specific classification object in white,
while the second one shows the non-classification objects as the background in black.
The two corresponding pictures of the black car classification are shown in fig. ~\ref{fig:label1}.
The two corresponding pictures of the red car classification are shown in fig. ~\ref{fig:label2}.
The two corresponding pictures of the white car classification are shown in fig. ~\ref{fig:label3}.
Because of the aim to find cars with ferrari potential, this classification is called
the ferrari potential problem as is explained in a later chapter.

\begin{figure}
\subfigure[black car classification]{\frame{\includegraphics[width=0.33\textwidth]{\imgRoot/schwarzes_Auto_weiss.png}}} \hfill
\subfigure[background classification]{\frame{\includegraphics[width=0.33\textwidth]{\imgRoot/schwarzes_Auto_schwarz.png}}}
\caption{reference images of black car classification}
\label{fig:label1}
\end{figure}
\begin{figure}
\subfigure[red car classification]{\frame{\includegraphics[width=0.33\textwidth]{\imgRoot/rotes_Auto_weiss.png}}}\hfill
\subfigure[background classification]{\frame{\includegraphics[width=0.33\textwidth]{\imgRoot/rotes_Auto_schwarz.png}}}
\caption{reference images of black car classification}
\label{fig:label2}
\end{figure}
\begin{figure}
\subfigure[white car classification]{\frame{\includegraphics[width=0.33\textwidth]{\imgRoot/weisses_Auto_weiss.png}}}\hfill
\subfigure[background classification]{\frame{\includegraphics[width=0.33\textwidth]{\imgRoot/weisses_Auto_schwarz.png}}}
\caption{reference images of black car classification}
\label{fig:label3}
\end{figure}

\subsection{Interim Test}
As an interim test, in fig. ~\ref{fig:label5} the results of the given normRGB(..) function are shown.

\begin{figure}
\subfigure[black car classification]{\frame{\includegraphics[width=0.33\textwidth]{\imgGeneratedRoot/result_ml_RGB_weiss.png}}}\hfill
\subfigure[red car classification]{\frame{\includegraphics[width=0.33\textwidth]{\imgGeneratedRoot/result_ml_RGB_rot.png}}}
\subfigure[white car classification]{\frame{\includegraphics[width=0.33\textwidth]{\imgGeneratedRoot/result_ml_RGB_schwarz.png}}}
\caption{Naive Bayes result on training image - RGB}
\label{fig:label5}
\end{figure}

In picture b) tje red parts of the car are detected quite well. Because of the black tyres in the red car reference image, the black car is mostly classified as a red car, too.
Corresponding parts of the white car are detected as well.
The effect of disturbances in the image can be seen by inspecting picture a): Due to the reflection of the sun on the back car, major parts of the floor are classified as being part of a black car.
Another source of malclassification which is caused by the pure color-based approach is illustrated by picture c): Matching parts of the other cars are detected, while the white car itself is not classified properly. This mismatch could also be caused by the very similar floor.

In a second test, the training image was used as a test image for the given hsv(..) function.

\begin{figure}
\subfigure[black car classification]{\frame{\includegraphics[width=0.33\textwidth]{\imgGeneratedRoot/result_ml_HSV_weiss.png}}}\hfill
\subfigure[red car classification]{\frame{\includegraphics[width=0.33\textwidth]{\imgGeneratedRoot/result_ml_HSV_rot.png}}}
\subfigure[white car classification]{\frame{\includegraphics[width=0.33\textwidth]{\imgGeneratedRoot/result_ml_HSV_schwarz.png}}}
\caption{Naive Bayes result on training image - HSV}
\label{fig:label6}
\end{figure}

These images give a further impression for the second feature calculation method.

As a third test, the training image was used as a test image for a combination of the normRGB(..) and hsv(..) functions. Unfortunately, this test could not be completed because of lack of RAM capacity.
The results of this test could have been interesting but the results of normRGB(..) were sufficient for continuation.

\subsection{Ferrari Potential Problem}
With the help of the Naive Bayes approach, the Ferrari potential of a given car shall be examined as a quantification of the similarity level between this car and a Ferrari. Thus, in this example the Ferrari potential of the red, black and white cars is analyzed.


The test image is now a picture of a real Ferrari and not longer the raw classification image (Fig. \ref{fig:\imgRoot/test_image1.jpeg}).

\image{\imgRoot/test_image1.jpeg}{0.8}{%
		Test picture (shows the target object - a real Ferrari).}

In this test image, a red Ferrari in front of white walls and floor can be seen.

\section{Results:}

As expected, only the red car classification classified the Ferrari correctly: As only the color is evaluated, the red color of the VW training car dominates the end result
(Fig.) \ref{fig:\imgGeneratedRoot/1rot_result_ml.png}).

\image{\imgGeneratedRoot/1rot_result_ml.png}{0.8}{%
		Naive Bayes result on test picture - RGB (based on red car classification).}

In this picture, one can see that the area of the ferrari
which was shown in fig. \ref{fig:\imgRoot/test_image1.jpeg}
has been nearly completly detected and correctly marked in red.
Apart from parts of the windshield, the background has been largely classified correctly.

As a comparison, the result of Naive Bayes based on the black and green car classification
is shown in fig. ~\ref{fig:label4}.

\begin{figure}
\subfigure[black car classification]{\frame{\includegraphics[width=0.49\textwidth]{\imgGeneratedRoot/1schwarz_result_ml.png}}} \hfill
\subfigure[green car classification]{\frame{\includegraphics[width=0.49\textwidth]{\imgGeneratedRoot/1weiss_result_ml.png}}}
\caption{Naive Bayes result on test picture - RGB}
\label{fig:label4}
\end{figure}

Here, one can see that the neither the black VW nor the green VW have a high Ferrari potential,
as they are not correctly colored in (foreground classification indicator) red in the pictures.

\section{Interpretation:}

Because of the fact that in our implementation only the normRGB(...) function was used,
in the present case the red car classification can be seen as Ferrari classification.
We have chosen a picture of a red ferrari because of the cliché that ferraris are red.
The only reason why the ferrari has been detected with help of Naive Bayes
has been the lucky situation that one of the three classification cars has had
a similar color as the Ferrari.
The direct conclusion from that is, that every picture of Ferraris differing in color, e.g. of a yellow one, would have had a similar bad
classification result concerning the red car reference picture as the black and white reference pictures had w.r.t. the shown red Ferrari.

\newpage
\section{Printed Code:}

%\lstset{language=<C++>}
%\begin{lstlisting}
%test
%\end{lstlisting}
\lstinputlisting[breaklines=true]{../native/aia4.cpp}

\end{document}
